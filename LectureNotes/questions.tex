\documentclass[10pt,a4paper]{article}
\usepackage[utf8]{inputenc}
\usepackage{amsmath}
\usepackage{amsfonts}
\usepackage{amssymb}
\usepackage{graphicx}
\usepackage[svgnames]{xcolor}
\usepackage{hyperref}
\definecolor{darkblue}{rgb}{0,0,.6}
\hypersetup{colorlinks=true, breaklinks=true, linkcolor=darkblue, menucolor=darkblue, urlcolor=darkblue, citecolor=darkblue}
\usepackage[framemethod=default]{mdframed}

\usepackage[]{natbib}


\mdfsetup{%
backgroundcolor=lightgray}


\author{Florian Hartig}
\title{Foundations of Bayesian Inference}
\date{Webinar Bayreuth, 31.4. 10.00 - 12.30}
\begin{document}
\maketitle
\begin{abstract}

This documents sketches the question I'll cover in the session. I hope that reading this document in advance will enable you to benefit better from the presentation because you had already time to reflect about the things we will cover.

\end{abstract}

\section{Bayesian inference}


\begin{mdframed}[frametitle={Ask yourself}]
\begin{itemize}
  \item What is "statistical inference"
  \item What is a (parametric) statistical model
  \item How do you get from a model to a likelihod
  \item What is Bayesian inference
  \item Is there a difference between "Bayesian models" and "normal statistical models"?
\end{itemize}
\end{mdframed}\vspace{0.3cm}

Suggested readings: any stats book, \citet{Ellison-Bayesianinferencein-2004}, Box 1 in \citep{Hartig-Statisticalinferencestochastic-2011} gives a super short overview as well.


\section{Differences between Bayes and Frequentism, p-values vs. posterior, etc.}

\begin{mdframed}[frametitle={Ask yourself}]
\begin{itemize}
  \item Why are frequentists called frequentists? What is it that frequentists tried to achieve when they developed this philosophy at a time when everyone was a Bayesian?
  \item Recall the definition of things frequentists report - p-value, MLE estimate and a frequentist confidence intervals. What is the difference in the definition of a frequentist CI and a Bayesian CI? 
\end{itemize}
\end{mdframed}\vspace{0.3cm}

For a discussion on frequentist thinking read ONLY section 4 of \citet{Efron-250yearargument-2013}.

\section{What's the practical difference, and why using Bayes}

\begin{mdframed}[frametitle={Ask yourself}]
\begin{itemize}
  \item Why do you want to use Bayesian inference, what are the advantages?
  \item What are the problems with p-values?
  \item What are the computational problems with calculating likelihoods for hierarchical models?
  \item Can you be both a Bayesian and a frequentist?
\end{itemize}
\end{mdframed}\vspace{0.3cm}

Suggested readings: \citep{Cohen-earthisround-1994, Kass-Statisticalinferencebig-2011}.

\section{Priors}

\begin{mdframed}[frametitle={Ask yourself}]
\begin{itemize}
  \item What is the difference between an informative and a uninformative / vague prior
  \item Why is a flat prior not neccesarily uninformative
  \item Why do priors change under parameter transoformation
\end{itemize}
\end{mdframed}\vspace{0.3cm}

I'm sorry that I don't have a more readable reference than \citet{Kass-selectionofprior-1996}, maybe \citet{Irony-Non-informativepriorsdo-1997}, but I fear you will find those all rather technical. 


\section{Software to do Bayesian inference}

\begin{mdframed}[frametitle={Ask yourself}]
\begin{itemize}
  \item Why does Bayesian inference usually use Markov-Chain samplers to calculate the posterior?
  \item Which samplers are there?
  \item Which software packages do you know? How would you classify them?
\end{itemize}
\end{mdframed}\vspace{0.3cm}

Suggested readings: a good but technical introduction on MCMC is \citet{Andrieu-introductiontoMCMC-2003}. A short overview of sampling algorithms can be found around Fig. 7 in \citet{Hartig-Statisticalinferencestochastic-2011}. About the software, contemplate about the following: there are three types of software for Bayesian inference

\begin{itemize}
\item Stand-alone samplers that work with a posterior function that has to be provided such as http://cran.r-project.org/web/packages/mcmc/index.html
\item General-purpose frameworks that allow specifying the model and then do the sampling, mainly OpenBugs,JAGS and STAN, all based on the bugs language.
\item Packages for specialized applications that allow only one model type, such as http://cran.r-project.org/web/packages/MCMCglmm/index.html 
\end{itemize}

See also http://cran.r-project.org/web/views/Bayesian.html

\section{Special topic - simulation-based inference}

\begin{mdframed}[frametitle={Ask yourself}]
\begin{itemize}
  \item Read the introduction of the reference to answer: why are there statistical model structures for which even standard Bayesian samplers such as Jags can not approximate the likelihood / posterior?
\end{itemize}
\end{mdframed}\vspace{0.3cm}

Suggested readings: \citep{Hartig-Statisticalinferencestochastic-2011}


\bibliographystyle{chicago}

\bibliography{/Users/Florian/Home/Bibliography/Databases/flo.bib}




\end{document}